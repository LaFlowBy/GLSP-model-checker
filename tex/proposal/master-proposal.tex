\documentclass[conference,onecolumn]{IEEEtran}

\usepackage[nolist]{acronym}
\usepackage[backend=bibtex]{biblatex}
\usepackage{graphicx}
\usepackage{hyperref}

\addbibresource{master-proposal.bib}

\begin{document}

  \title{Proposal: Web-based Model Checking UI with GLSP}

  \author{\IEEEauthorblockN{Florian Weidner}
    \IEEEauthorblockA{Philipps-University Marburg, Germany\\
      Department of Mathematics and Computer Science, Software engineering group\\
      May 06, 2025\\
  }}

  \maketitle

  \IEEEpeerreviewmaketitle

  \section{Introduction}
  \label{sec:introduction}
  The goal of this project is to create a user interface for a model checking tool using the \ac{glsp} framework. 

  \section{Background}
  \label{sec:background}
  Theoretic Background and Frameworks used.

  \subsection{Model Checking}
  \label{subsec:model-checking}
  Model checking is an automatic technique for verifying finite-state reactive systems, such as sequential circuit designs and communication protocols.
  \cite{modelchecking1} \cite{modelchecking2}

  \subsection{\ac{emf}}
  \label{subsec:emf}

  The EMF project is a modeling framework and code generation facility for building tools and other applications based on a structured data model.
  \cite{emf} \cite{emf-repo}

  \subsection{\ac{glsp}}
  \label{subsec:glsp}

  The \ac{glsp} is a framework that allows developers to create graphical user interfaces for web-based diagram editors.
  \cite{glsp-repo}

  \subsection{Henshin}
  \label{subsec:henshin}
  Henshin is an in-place model transformation language for the Eclipse Modeling Framework (EMF).
  \cite{henshin-repo}

  \section{Related Work/Software}
  \label{sec:related-work}

  Similar already existing tools for model checking: Henshin, Groovy...

  \section{Project Plan}
  \label{sec:project-plan}

  Things that need to be done:

  \begin{itemize}
    \item GLSP Editor for the metamodel (.ecore)
    \item GLSP Rule Editor for the transformation rules (.henshin)
    \item GLSP Editor for creating instance models and applying rules to the instance model (.xmi)
    \item Option to start the model checking process
    \item Connect Rules and Instance Models to the metamodel
  \end{itemize}

  \section{Scientific Contribution}
  \label{sec:scientific-contribution}

  Comparison of the existing tools and the new tool.

  \section{Open Questions}

  \begin{itemize}
    \item Was soll alles ins Proposal??
    \item GLSP Vorlagen?
    \item Für welche Clients soll die UI sein (theia, eclipse, vscode, electron)?
    \item Hensin Sdk Dokumentation?
    \item How to use Henshin SDK in the GLSP application?
    \item Repository Gitlab oder Github?
    \item Welchen Umfang hat die Arbeit?

  \end{itemize}

  \section{Meeting Notes}

  \begin{itemize}
    \item erster schritt anwendung der regeln auf instance Models
    \item schreiben über grundlagen -> eclipse an sich
    \item henshin interpreter api
    \item 2 stufen: instance transformationen -> regeln spezifizieren
  \end{itemize}


  \printbibliography

  \begin{acronym}
    \acro{glsp}[GLSP]{Graphical Language Server Platform}
    \acro{emf}[EMF]{Eclipse Modeling Framework}

  \end{acronym}

\end{document}
